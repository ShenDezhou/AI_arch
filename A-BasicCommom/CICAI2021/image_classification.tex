% This is samplepaper.tex, a sample chapter demonstrating the
% LLNCS macro package for Springer Computer Science proceedings;
% Version 2.20 of 2017/10/04
%
\documentclass[runningheads]{llncs}
%
\usepackage{comment}
\usepackage{graphicx}
\usepackage[breaklinks=true]{hyperref}
% Used for displaying a sample figure. If possible, figure files should
% be included in EPS format.
%
% If you use the hyperref package, please uncomment the following line
% to display URLs in blue roman font according to Springer's eBook style:
% \renewcommand\UrlFont{\color{blue}\rmfamily}

\begin{document}
%
\title{Enhance Image Classification Performance}

% INITIAL SUBMISSION 
\def\YOFOSubNumber{98}  % Insert your submission number here
%\begin{comment}
\titlerunning{CICAI2021 submission ID \YOFOSubNumber} 
\authorrunning{CICAI2021 submission ID \YOFOSubNumber} 
\author{Anonymous CICAI submission}
\institute{Paper ID \YOFOSubNumber}
%\end{comment}
%******************

% CAMERA READY SUBMISSION
\begin{comment}
\titlerunning{Abbreviated paper title}
% If the paper title is too long for the running head, you can set
% an abbreviated paper title here
%
\author{First Author\inst{1}\orcidID{0000-1111-2222-3333} \and
Second Author\inst{2,3}\orcidID{1111-2222-3333-4444} \and
Third Author\inst{3}\orcidID{2222--3333-4444-5555}}
%
\authorrunning{F. Author et al.}
% First names are abbreviated in the running head.
% If there are more than two authors, 'et al.' is used.
%
\institute{Princeton University, Princeton NJ 08544, USA \and
Springer Heidelberg, Tiergartenstr. 17, 69121 Heidelberg, Germany
\email{lncs@springer.com}\\
\url{http://www.springer.com/gp/computer-science/lncs} \and
ABC Institute, Rupert-Karls-University Heidelberg, Heidelberg, Germany\\
\email{\{abc,lncs\}@uni-heidelberg.de}}
\end{comment}
%******************
\maketitle              % typeset the header of the contribution

\begin{abstract}

  The \underline{B}idirectional \underline{E}ncoder \underline{R}epresentations from \underline{T}ransformers represents the recent popularity of many tasks in natural language processing.
  These pre-trained language models have been used to enhance computer vision tasks for many years.
  This paper uses a pre-trained transformer model to encode image pixels, and then connects to the fully connection layer, dropout layer and Softmax layer.
  The performance of the classification model is better than the current best iGPT-L model.
  It surpasses the current best accuracy in the CIFAR-10/100 image set and the COVID-19 tomographic image classification set.

\keywords{Image Classification  \and Transformers \and Attention.}
\end{abstract}
%
%
%

%%%%%%%%% BODY TEXT
\section{Introduction}

  Unsupervised pre-training is important to modern research of deep learning.
  Lee et al.~\cite{lee2009convolutional} used the pre-training approaches in Computer Vision tasks in 2009,
  and, Nair and Hinton~\cite{nair2010rectified} proved that the pre-training process is supplementary, which can be omitted in some cases, in the Computer Vision tasks later from 2010 to 2016,
  However, it started to flourish since Mikolov et al.~\cite{mikolov2013distributed} had proposed Word2Vec in the natural language processing,
  recently, Devin et al.~\cite{devlin2019bert}'s BERT language model dominates in most tasks in natural language processing.
  The pre-training approach in natural language processing is close resemble to Vicent et al.~\cite{vincent2008extracting}'s Denoising Autoencoder model initially designed for images processing.
  Thus, the pre-training step becomes important in CV and NLP.

%-------------------------------------------------------------------------

\section{Recent Work}

  Chen et al.~\cite{chen2020generative} trained image representation by sequence Transformers and tested on CIFAR-10 to show it is outperforming to Wide-ResNet
which was inspired by unsupervised natural language representation learning.
  Wang et al.~\cite{wang2019development} reviewed that convolutional neural networks had been proposed in the 1960s, and had its implementation in the 1980s,
and until LeCun et al.~\cite{lecun1990handwritten}'s first experiment on handwritten digit recognition, CNN's great potential had been revealed.
  In the 2010s, Krizhevsky et al.~\cite{krizhevsky2012imagenet} proposed the deep architecture, AlexNet, by concatenating multiple components of CNN layers.
Several years later, a lot of variants of AlexNet had been proposed by researchers and the accuracy of ImageNet had been greatly improved, e.g. ZFNet~\cite{zeiler2014visualizing}, VGG~\cite{simonyan2014very}, GoogLeNet~\cite{szegedy2015going}, ResNet~\cite{he2016deep},
  ResNeXt~\cite{xie2017aggregated}, inception-ResNet-v2~\cite{szegedy2016inception}, DenseNet~\cite{huang2016deep}.
  Lu and Weng~\cite{lu2007survey} concluded that for the multi-source image classification tasks, additional information such as signatures, texture, context, and ancillary data can be combined to achieve better performance.
And it is difficult in handling the dichotomy between pixels and natural language texts in a single model. For per-pixel approach, non-parametric classifiers, e.g. neural networks, support vector machines, and decision trees,
are the most well-known algorithms for their performance and generalization advantages in the late 1990s and 2000s.
  Cui et al.~\cite{cui2020revisiting} proposed several whole-word-masking pre-trained Chinese language models,
which are improved versions of BERT~\cite{devlin2019bert} pre-trained language models, namely RoBERTa(3), RoBERTa-L(3), and RoBERTa-L(24). The training of the RoBERTa(3) language model was based on Devlin et. al.'s BERT model,
moreover, the pre-trained Chinese language models use extra 5 billion training tokens comparing to 0.4 billion tokens in the BERT model.
These models achieved better performance in Chinese machine reading comprehension, Chinese document classification, and other downstream natural language tasks.
  He and Peng~\cite{he2017fine} combined the vision stream and language stream as two parallel channels for extracting multi-source information in the image classification task,
and tested on the CUB-200-2011 image dataset and achieved $85.55\%$ by combining GoogLeNet~\cite{szegedy2015going} and CNN-RNN~\cite{reed2016learning}, the result outperformed many competitors.
  Fern{\'a}ndez et al.~\cite{fernandez2014we} compared different classifiers in small datasets, and they found that the random forest algorithm ranks first among the 179 classifiers.
  Recall that Kim~\cite{kim2014convolutional} had proved that unsupervised pre-trained language model $word2vec$ and CNN outperformed many other machine learning algorithms,
e.g. support vector machines and conditional random field, in many datasets such as movie reviews, Stanford sentiment treebank, and TREC question.
Cui et al.'s pre-trained language model, namely RoBERTa(3), RoBERTa-L(3), and RoBERTa-L(24), had improved performances over many other machine learning algorithms.
BERT and RoBERTa-L(24) models both achieved an f1-score of 97.8\% in the THUCNews dataset, which contains 65,000 news reports in 10 domains.

%-------------------------------------------------------------------------

\section{Image Classification by Convolutional and Transformer Networks}

We use the Jittor\cite{hu2020jittor} library as the modeling tool, and the TsinghuaDogs\cite{Zou2020TsinghuaDogs} dataset to compare the performance with competitors.The experiment was performed on an Nvidia V100, with 16GB of RAM,
which can provide 100 tera-flops of computation capability.

\subsection{Convolutional Neural Networks for Image Classification}

\par Three of the most popular approaches for image classification tasks are per-pixel, subpixel, and heterogeneous.
CNN models are fundamental and often used as baselines to compare the classification performance.
The vision system of human has a mechanism that focuses limited attention on the key information to save resources and to get the most efficient information quickly.
Before introduction of the Attention mechanism, there was a problem in RNN that long distances of information would be weakened,
just as people with weak memory could not remember the past.

\par Image pre-process steps including resizing to $256*256$, center-cropping by $224*224$, a random horizontal flip, conversion to (0,1), and normalization to 0.485 of mean value and 0.229 of standard error.

\par The ResNet has been proved to be efficient and effective, thus, a lot of time could be saved by starting from a pre-trained model for a new Computer Vision task.
It is not surprising that the residual mechanism can learn deep layers with some layers skipped for different tasks.


\begin{table}
\begin{center}
\begin{tabular}{|l|r|}
\hline
Model   &  Accuracy \\
\hline
\hline
AlexNet & 41.73\%\\
GoogLeNet & 68.84\%\\
MobileNet\_v2 & 63.65\%\\
ShuffleNet\_v2\_x0\_5  &	45.38\% \\
Attention-ResNet152-4H-256L &  	55.00\% \\
Attention-ResNet152-1H-1024L &  	67.57\% \\
\hline
\hline
Attention-ResNet152-1H-1024L &  	78.38\% \\
DenseNet121   	&	75.29\% \\
SE-ResNet50 & 79.47\% \\
ResNet50 & 82.50\%\\
Gelu-ResNet152 &	64.06\% \\
ResNet152  	&	85.14\% \\
SE-ResNet152  	&	83.34\% \\
SE-ResNet152+ResNet152 & 82.62\% \\
SE-ResNet152+AlexNet & 80.38\% \\
ResNet200  	&	73.19\% \\
\hline
\hline
ResNet152  	&	84.87\% \\
SE-ResNet152  	&	82.92\% \\
\hline
\end{tabular}
\end{center}
\caption{Comparision of model accuracy on the TsinghuaDogs dataset. The top part models use 10\% of the whole TsinghuaDogs data, the middle part models use 80\% of the TsinghuaDogs data,
and the bottom part models use 100\% of the TsinghuaDogs data.}
\end{table}

\subsection{Dataset Amount Usage}

From the Table (1) ResNet152 and SE-ResNet152 in the middle part and the bottom part, we can see that there is imbalance on the dataset for different classes, the top class has 7449 instances,
and the bottom class has 200 instances. As a result, using 80\% of the TsinghuaDogs data might beat the 100\% under Stochastic Gradient Descent optimizer.
In later discussion, we use 80\% of the TsinghuaDogs data to train the model for best performance and accuracy.

\subsection{Model Description}

All models in the Table (1) share a common architecture that a base feature selection layer, such as AlexNet, followed by a fully connected layer which was used as a mapping function from ImageNet dataset classes to TsinghuaDogs dataset classes.
The Attention based model use two $1*1$ convolutional layers serves as transition between the average-pool layer and the fully connected layer.
Between that we use the Multi-head Attention layer with 1024 as the max sequence and 1 head, the k-dim and v-dim are 1024 and the self attention mechanism was used.
The AlexNet, GoogLeNet, MobileNet, ShuffleNet, DenseNet, ResNet, SE-ResNet are the well-known models. The Gelu-ResNet replace the Relu with Gelu function.
The ResNet200 increase the third phase parameter from 36 to 48, with a parameter of $[3,8,48,3]$.
The Attention-ResNet152-4H-256L and Attention-ResNet152-1H-1024L model differs only on the number of heads, and the input dimension, the first model has 4 heads and 256 dimension for embedding dimension, K-dim and V-dim.
For the Attention layer, the shape of the input was transformed to (batch, channel, sequence), then transposed to (batch, sequence, channel) for the Attention layer.
After the Attention operation, the shape of the image was transposed back to (batch, channel, sequence), and then for the rest of the operations.

\subsection{Discussion}

From the Table (1) we can see that the Attention layer doesn't improve the performance on the TsinghuaDogs validation dataset, there is 6.85\% gap comparing to the ResNet152 model.
Experiment on the Attention model number of heads and sequence length showed that, (Attention-ResNet152-4H-256L \&  	55.00\%, Attention-ResNet152-1H-1024L \&  	67.57\%),
increasing the number of head and reducing the sequence length have negative impact on the accuracy of the TsinghuaDogs Dataset.
Experiment showed that, the Attention mechanism following the ResNet doesn't show improvement on the accuracy, and so as the Squeeze-Extract mechanism, Gelu activation, and combination of ResNet with other networks, such as AlexNet or SEResNet.
\par To state it clear that, by adding the Transformer layer around the ResNet doesn't improve the overall performance of the image classification task.
Careful architecture design should be performed, otherwise, there would be negative gain from the architecture redesign.
Experiment showed that the ResNet152 is still a strong baseline for the Image Classification task.

%-------------------------------------------------------------------------
\section{Image Classification via Pre-trained Transformers Language Models}

\subsection{Approach}

Our approach consists of a pre-training stage followed by a fine-tuning stage.
In the pre-training, we use the BERT objectives, and the sequence Transformer architecture to predict language tokens.

\par Given an unlabeled dataset $ X$, the BERT objective samples a sub-sequence $ S \in \{C\}$,
$C$ represents all possible tokens, and such that for each index $i \in  S$,
there is independent probability $15\%$ of appearing in $ S$,
name the token $M$ as the BERT mask.
As equation (1), train the language model by minimizing the BERT objective of the "masked" elements $x_M$
conditioned on the "unmasked" ones $x_{\left[1,n\right]\backslash M}$.

\begin{equation}
  \lambda = \mathop{ E}\limits_{x\sim X} \mathop{ E}\limits_{M} \sum_{i\in S}{\bigl[-\log{p(x_i\mid x_{\left[1,n \right]\backslash M})\bigr]}}
\end{equation}

\par The transformer decoder takes the image pixels and meta characters sequence $x_1,\cdots,x_n$ and produces a $d$-dimensional
embedding for each position.
Then we use a fully connected layer as a non-linear function from embeddings to image classes.
The dropout layer and the Softmax layer are used for better transfer performance between the training and the test datasets.

\subsection{The Power of the Transformer Encoder on the Document Classification}

Generality speaking, the performance of the pre-trained language models achieved in the document classification tasks.

We used the RoBERTa-L with 3 and 24 layers as encoder, and tested on two Chinese Judicial datasets, which has two and four classes.
The \textit{Two-Case} dataset annotated civil and criminal cases, and the \textit{Four-Case} dataset annotated by civil, criminal, intellectual property, and administrative cases.
The \textit{Two-Case} dataset has $19,508$ training documents and $2,000$ test documents, and the \textit{Four-Case} dataset has $34,789$ training documents and $2,013$ test documents.

From Table (2), by combining the pre-trained language model with a fully connected layer as the document classification model,
the test accuracy exceeds the other popular machine learning algorithms.
Therefore, the pixel channels of an image can be properly represented by these pre-trained language models.

\begin{table}
\begin{center}
\begin{tabular}{|l|c|r|}
\hline
Language Model 	& Dataset & Accuracy \\
\hline\hline
RoBERTa-L(24) & Two-Case	&	99.80\% \\
RoBERTa-L(24) & Four-Case	&	99.76\% \\
RoBERTa-L(3)  & Four-Case	&	96.35\% \\
\hline
\end{tabular}
\end{center}
\caption{Comparision of accuracy of the pre-trained language models on different datasets in the document classification task.}
\end{table}
%-------------------------------------------------------------------------


\subsection{Image Classification Model with Transformer Encoder}


For the per-pixel image classification approach, for every RGB channel of pixels in an image,
each pixel had its pixel-channel code, ranging from 0x00 to 0xff for different colors.
Thus, taking these pixels in an image is resembled to ASCII characters in a document.
An experiment show that document classification capability can be transferred to image classification naturally.

\begin{figure*}
\begin{center}
  \fbox{\rule{0pt}{2in} \includegraphics[width=0.9\linewidth]{classification.pdf}}
\end{center}
   \caption{Concatenation, encoder, representation, and extraction layers for image classification task.}
\label{fig:short}
\end{figure*}

We design simple classification models without too many layers, as Figure (1) shows,
and use the CIFAR-10 and CIFAR-100 dataset as an example to show the architecture of the model.
The model architecture has seven functional layers:

\begin{itemize}
\item {\bf Input layer}
\item {\bf Concatenation layer}
\item {\bf Trim layer}
\item {\bf Encoder layer}
\item {\bf Embedding layer}
\item {\bf Feature Extraction layer}
\item {\bf Output layer}
\end{itemize}


\par In total, 349 computed tomography images were collected from the COVID-19 reports in the MedPix, LUNA, Radiopaedia,
PubMed Central databases, which sizes range from $1637\times1225$ to $148\times61$.
The COVID19 lung tomography dataset contains 251 covid images and 291 covid negative images in the training set,
and contains 98 covid image and 100 covid negative images in the test set.
All images are resized to a resolution of $32\times32$ pixels, the RGB channels are kept and the alpha channel values are discarded.
For the COVID-19 lung tomography dataset, the metadata is the filename of the images, which can be treated as the supervised description of the image.
Peek some the filenames, e.g. \textit{covid\_201\_1.png}, \textit{covid\_201\_2.png}, and \textit{noncovid\_001\_1.png},


\par The CIFAR-10 dataset contains 60,000 color images with a resolution of $32\times32$ in 10 classes, and the CIFAR-100 dataset has 100 classes containing 600 images each.
Encode the image by the sequence of RGB channel values, in the order of Red-channel, Green-channel, and Blue-channel,
then encode other meta-data if provided, as a sequence of ASCII characters.
In the Concatenation layer, the pixel-channel value and metadata are concatenated, put a special token of $\textbf{[CLS]}$ at the start,
and put a $\textbf{[SEP]}$ token between channel values and metadata, put a special token of $\textbf{[SEP]}$ at the end.
In the Trim layer, due to the limit on the max sequence of the BERT language model, a sequence larger than $512$ needs to be trimmed before sending it to the BERT model.
Keep the first 256 characters and last 256 characters of the concatenated sequence, trimmed result contains the first 255 red-channel value,
some blue-channel value, and all the meta value in common cases.
In the Encoder layer and the Embedding layer, trimmed sequence of values are encoded by BERT-like models,
and get the encoded representation of the token $\textbf{[CLS]}$ as the images' language model embeddings.
In the Feature-Extraction layer, a combination of one dropout layer, one fully connected layer, and one softmax layer, as Equation (2), is used.
The Softmax layer is an approach that make the probability $z$ of each class more focused by taking exponent of it, then use normalization for each class probability.
In the Output layer, the classification label of the image is fed into the model.

\begin{equation}
  \sigma(\textbf{z})_j = {e^{\textbf{z}_j} \over {\sum_{k=1}^{\textit{K}} e^{\textbf{z}_k}}}, j=1,\cdots,\textit{K}.
\end{equation}

\par It is intuitive for the model to use both the channel values and the metadata in the training phase.
For the CIFAR-10 and CIFAR-100 dataset, the metadata is the filename of the images, which can be treated as the supervised description of the image.
Peek some the filenames, e.g. \textit{wagon\_s\_001343.png}, \textit{banana\_boat\_s\_001615.png}, and \textit{delivery\_truck\_s\_001529.png},
and it is beneficial for the classification model to understand the supervised descriptive information for the image.

%-------------------------------------------------------------------------


\section{Experiments and Results}


\subsection{Training}

\begin{figure*}
\begin{center}
  \fbox{\rule{0pt}{0in} \includegraphics[width=0.9\linewidth]{image-classification(3).pdf}}
\end{center}
   \caption{Accuracy of the image classification models with the pre-trained language encoder on the CIFAR-10 and CIFAR-100 dataset in the training epochs.}
\label{fig:short}
\end{figure*}



We use a batch size of 512 and train for $6\sim16$ epochs using AdamW with $\beta_1=0.99$, $\beta_2=0.999$, and weight decay of $10^{-8}$.
We set the learning rate to $10^{-4}$ and no warm-up or decay are used.
The dropout rate is $0.05$.

\par When fine-tuning, as Figure (2), we check the accuracy of the trained model on the test set and stopped training when it converges on $10^{-3}\sim10^{-4}$.

\par The experiment was performed on a Google Cloud TPU v3, with 32GB of RAM, and 8 chips with 16GB of the high speed of memory each,
which can provide 420 tera-flops of computation capability.

\subsection{Results}

Using the proposed model, we tried different pre-trained language models to see the impact on classification accuracy.
From the Table (3), for the same size of a dataset with larger classes, it needs more epochs and training time for the classification model.
For the same RoBERTa language model with different numbers of transformer layers, 24 layers had better accuracy than 3 layers,
however, the training cost grows for the larger language model.
For the same fine-tuned language model, class number has positive impact on the accuracy.
And the fewer classes the dataset has, the more accurate results the model can achieve.

\par RoBERTa-S is short for RoBERTa-Small which has 768 hidden dimension, RoBERTa-L is short for RoBERTa-Large  which has 1024 hidden dimension.

\begin{table}
\begin{center}
\begin{tabular}{|l|c|c|r|}
\hline
Language Model 	& Dataset  & Accuracy \\
\hline
\hline
RoBERTa-S(3)   & COVID19 	&	100.0\% \\
RoBERTa-L(3)  & COVID19 	&	99.51\% \\
RoBERTa-L(24)  & COVID19 	&	100.0\% \\
\hline
RoBERTa-S(3)   & CIFAR-10 	&	99.74\% \\
RoBERTa-L(3)  & CIFAR-10 	&	99.72\% \\
RoBERTa-L(24)  & CIFAR-10 	&	99.60\% \\
\hline
RoBERTa-S(3)    & CIFAR-100 	&	99.66\% \\
RoBERTa-L(3)  & CIFAR-100   & 	99.10\% \\
RoBERTa-L(24) & CIFAR-100  &	99.76\% \\
\hline\hline
iGPT-L & CIFAR-10  & 99.0\% \\
iGPT-L & CIFAR-100 & 88.5\% \\
\hline
\end{tabular}
\end{center}
\caption{Comparision of accuracy of the pre-trained language models and iGPT-L on the COVID19, CIFAR-10 and CIFAR-100 datasets in the image classification task.
    (a) iGPT-L model was trained far more than 18 epochs, it was trained by 1,000,000 iterations of pre-training and a few epochs for fine-tuning.}
\end{table}


\subsection{Discussion}

The iGPT-L was trained on the ImageNet dataset, and it has 48 layers of transformers and 1.4 billion parameters, and the embedding size takes 1536,
so it has almost identical layers with GPT-2 language model, except for 1.5 billion parameters and 1600 embedding size.
Compare to iGPT-L's accuracy of 96.3\% on the CIFAR-10 dataset without augmentation and 82.8\% on the CIFAR-100 dataset, our models have preferable better results.
The reason that the model's outstanding performance lies in the large pre-trained data for BERT, on top of that fine-tuned by RoBERTa, and use of extra language corpus of
$5.4$ billions of tokens of wiki data and other resources.
The transformers in the pre-trained language models use multiple layers for representing images and may be used in other Computer Vision task, e.g. object detection, gesture prediction.

\section{Reproducibility}

The convolutional neural networks for image classification experiment notebook can be found at https://colab.research.google.com/drive/\\
167uMdDGmXSr1xwdM4GFwfiuBpz0bCbeA?usp=sharing. Related code are placed on https://github.com/ShenDezhou/CAIL2021/tree/master/dogflb.

The image classification model with transformer encoder experiment notebook can be found at https://colab.research.google.com/drive/\\
1saMYaaR3ih1eo9ttYy0Ncb5V003uVyEV?usp=sharing. Related code are placed on https://github.com/ShenDezhou/CAIL/tree/master/CAIL2020/rlfl.



\section{Conclusion}

This article proposes a new idea of using pre-trained language models for image characterization and image classification.
Through empirical analysis, the negative impact of the transformer on the CNN model is discussed, and it is proved that the pre-trained language model can improve the performance in image classification.
This finding may benefit various topics of computer vision tasks.
The test shows that the proposed model is better than the classification effect of the iGPT-L model without enhancing the image data set.
The accuracy is $99.60\%\sim99.74\%$ on the CIFAR-10 image set, $99.10\%\sim99.76\%$ on the CIFAR-100 image set, and $99.51\%\sim100.00\%$ on the COVID-19 tomographic image test set.

%
% ---- Bibliography ----
%
% BibTeX users should specify bibliography style 'splncs04'.
% References will then be sorted and formatted in the correct style.
%
% \bibliographystyle{splncs04}
% \bibliography{mybibliography}
%
{\small
\bibliographystyle{splncs04}
\bibliography{cpvr_classification}

}

\end{document}
