% This version of CVPR template is provided by Ming-Ming Cheng.
% Please leave an issue if you found a bug:
% https://github.com/MCG-NKU/CVPR_Template.

\documentclass[review]{cvpr}
%\documentclass[final]{cvpr}

\usepackage{times}
\usepackage{epsfig}
\usepackage{graphicx}
\usepackage{amsmath}
\usepackage{amssymb}
\usepackage[linesnumbered,lined,boxed,commentsnumbered]{algorithm2e}
% Include other packages here, before hyperref.


% If you comment hyperref and then uncomment it, you should delete
% egpaper.aux before re-running latex.  (Or just hit 'q' on the first latex
% run, let it finish, and you should be clear).
\usepackage[pagebackref=true,breaklinks=true,colorlinks,bookmarks=false]{hyperref}


\def\cvprPaperID{8182} % *** Enter the CVPR Paper ID here
\def\confYear{CVPR 2021}
%\setcounter{page}{4321} % For final version only


\begin{document}

%%%%%%%%% TITLE
\title{Text Detection and Chinese Character Recognition in Natural Scene Images via Pre-trained Vision and Language Models}

\author{Dezhou Shen\\
Department of Computer Science\\
Tsinghua University\\
Beijing, CN 100084\\
{\tt\small sdz15@mails.tsinghua.edu.cn}
% For a paper whose authors are all at the same institution,
% omit the following lines up until the closing ``}''.
% Additional authors and addresses can be added with ``\and'',
% just like the second author.
% To save space, use either the email address or home page, not both
}

\maketitle


%%%%%%%%% ABSTRACT
\begin{abstract}

  Detecting and reading text from natural scene images is an essential computer vision tasks.
  Giving the fact that Chinese characters have more tokens than English characters, OCR in Chinese is much harder than English.
  I encode high dimension features extracting from images by ResNet followed by the pre-trained language models of
  \underline{B}idirectional \underline{E}ncoder \underline{R}epresentations from \underline{T}ransformers,
  then connect to a recurrent neural network layer, and a fully connected layer in the Character Recognize architecture.
  The proposed recognition model outperforms the ResNet recognition model, and achieved accuracy of $79.82\%$ on the Baidu2019 street scene test set.

\end{abstract}

%%%%%%%%% BODY TEXT
\section{Introduction}

  Reading text from photographs is essential to Natural language processing and Computer Vision.
  OCR of printed documents and recognition of hand-written digits are already known as solved problems,
  however, recognizing characters in natural scenes is far more difficult.
  Regarding to the fact that Chinese language has more characters than English, thus it is more difficult to get a satisfactory model.
  Most researchers had laid lack of attention on the relevance between the natural language processing and the Computer Vision.
  In this paper, I discussed a novel method by unifying the representation of languages and visions in the representation layer.

  Unsupervised pre-training is important to the modern research in deep learning.
  Lee \etal~\cite{lee2009convolutional} used the pre-training approaches in Computer Vision tasks in 2009, and later from 2010 to 2016,
  Nair and Hinton~\cite{nair2010rectified} proved that the pre-training process is supplementary in the Computer Vision tasks,
  thus, can be omitted in some cases.
  However, it started to flourish in the natural language processing domain since Mikolov \etal~\cite{mikolov2013distributed} had proposed Word2Vec for the word representation.
  Not long after, Devin \etal~\cite{devlin2019bert}'s Bidirectional Encoder Representations from Transformers language model dominates
  in most frequent tasks in the natural language processing domain,
  which is close resemble of Vicent \etal~\cite{vincent2008extracting}'s Denoising Autoencoder model, which was initially designed for images.
  Therefore, the pre-training process becomes one of the most important procedures in deep learning.

%-------------------------------------------------------------------------

\section{Recent Work}

\begin{figure*}
\begin{center}
  \fbox{\rule{0pt}{2in} \includegraphics[width=0.9\linewidth]{detectionnet.pdf}}
\end{center}
   \caption{Combination of the ResNet, Feature Pyramid Net and Differentiable Binarization Net for scene text detection task.}
\label{fig:short}
\end{figure*}

\subsection{Detection}

  Wang \etal~\cite{wang2012end} used the convolutional neural networks as a detection model and achieved accuracy of 84\% by word-level on the ICDAR2003~\cite{lucas2003icdar} test dataset and Street View Text~\cite{wang2011end} dataset.
  LeCun \etal~\cite{lecun1990handwritten} started their first experiment on handwritten digit recognition, convolutional neural networks' great potential had been revealed since.
  Liao \etal~\cite{liao2020real} proposed a Differentiable Binarization module for the segmentation task in the text detection task, and achieved $82.8\%\sim84.9\%$ of F1-score on the TD500~\cite{yao2012detecting} dataset
   and $81.0\%\sim83.4\%$ on the CTW1500~\cite{liu2019curved} dataset.

\subsection{Recognition}

  Chen \etal~\cite{chen2020generative} trained the image representation using sequence Transformers and tested on the CIFAR-10 to show its outperforming to Wide-ResNet,
which was inspired by unsupervised natural language representation learning.
  Wang \etal~\cite{wang2019development} reviewed that convolutional neural networks had been first proposed in the 1960s, and had its implementation in the 1980s.
  In the 2010s, Krizhevsky \etal~\cite{krizhevsky2012imagenet} proposed the deep architecture, AlexNet, by concatenating multiple components of convolutional neural network layers.
  Netzer \etal~\cite{netzer2011reading} used unsupervised feature learning K-means to classify the digits cropped from Street View images and achieved accuracy of 90.6\% in the test dataset.
  A lot of variants of AlexNet had been proposed by researchers and the accuracy of ImageNet had been greatly improved, \eg ZFNet~\cite{zeiler2014visualizing}, VGG~\cite{simonyan2014very}, GoogLeNet~\cite{szegedy2015going}, ResNet~\cite{he2016deep},
  ResNeXt~\cite{xie2017aggregated}, inception-ResNet-v2~\cite{szegedy2016inception}, DenseNet~\cite{huang2016deep}.
  Lin \etal~\cite{lin2017feature} proposed multi-scale convolutional network in feature extraction, and the average precision on the COCO dataset is 56.9\%.

\subsection{Representation of Images}

  Lu and Weng~\cite{lu2007survey} concluded that for the multi-source image classification tasks, additional information such as signatures, texture, context, and ancillary data can be combined to achieve better performance.
And it is difficult in handling the dichotomy between pixels and natural language texts in a single model.
  Cui \etal~\cite{cui2020revisiting} proposed several whole-word-masking pre-trained Chinese language models,
which are improved versions of BERT~\cite{devlin2019bert} pre-trained language models, namely RBT3, RBTL3, and RoBERTa-wwm-ext-large.
These models achieved better performance in Chinese machine reading comprehension, Chinese document classification, and other downstream natural language tasks.
  He and Peng~\cite{he2017fine} combined the vision stream and language stream as two parallel channels for extracting multi-source information in the image classification task,
and tested on the CUB-200-2011 image dataset and achieved $85.55\%$ by combining GoogLeNet~\cite{szegedy2015going} and CNN-RNN~\cite{reed2016learning}, the result outperformed many competitors.

%-------------------------------------------------------------------------

\section{Text Detection Models}



  Recent scene text detection methods can be classified into Regression-based methods and Segmentation-based methods.
  By directly regressing the boxes of the text instances, Xie \etal~\cite{xie2019scene} proposed a dimension-decomposition region to handle scale problem in scene text detection.
  By combining pixel-level prediction and post-precessing, Tian \etal~\cite{tian2019learning} proposed pixel embeddings to clusters for the segmentation results.

\subsection{Image Pre-processing}

  During the training phase, the region of interest in the images will be augmented by left-right flipping, image affine, size resize, random crop and border shrinks.
  These augmentation makes the model more robust and transferable from training dataset to test dataset.

\subsection{Model Architecture}

  As Figure (1) shows, I combined the Residual Net, Feature Pyramid Network, and Differentiable Binarization network for text detection task.
  The model has five important layers, such as: Input layer, Residual Convolutional layer, Feature Pyramid layer, Differentiable Binarization layer and Output layer.
  The Residual Convolutional layer extract import high-level features from the image,
  and the Feature Pyramid layer join the four parts of features with different size of filter kernels, such as $[64,128,256,512]$.
  The Differentiable Binarization layer combines the features and use different mapping functions to find the power of the combination of features.
%-------------------------------------------------------------------------

\section{Text Character Recognition Models}

  Image are segmented by the detection model, then each segmentation is preprocessed to a fix size of $120\times32$.
The recognition model takes the adjusted image as input, and predict the sequence of Chinese Character dictionary index as output.


\begin{figure*}
\begin{center}
  \fbox{\rule{0pt}{2in} \includegraphics[width=0.9\linewidth]{resbertnet.pdf}}
\end{center}
   \caption{Combination of the Residual Net, Bert, Long-Short Term Memory and Fully Connection Net for Chinese character recognition task.}
\label{fig:short}
\end{figure*}

%-------------------------------------------------------------------------
\subsection{Model Architecture}

 As Figure (2) shows, I use a combination of Residual Net, Bert, Long-Short Term Memory and Fully Connection Net for character recognition task.

The model architecture has five functional layers:

\begin{itemize}
\item {\bf Input layer}
\item {\bf Residual Bert layer}
\item {\bf Sequence Decode layer}
\item {\bf Fully Connected layer}
\item {\bf Output layer}
\end{itemize}

All images are resized to a resolution of $120\times32$ pixels.
Encode the image by the sequence of RGB channel values, in the order of Red-channel, Green-channel, and Blue-channel.
In the Residual Bert  layer, the pixel-channel value are extracted by the residual convolutional network,
then fed into the Bert language model to encode each sub-pixels.
In the Sequence Decode layer, two layers of Long-Short Term Memory networks to decode each Chinese character meanings.
In the Fully Connected layer, extracted tokens are connected to the output layer.
In the Output layer, the encoded Chinese character index sequence are feed into the model.


\subsection{The Sub-pixel Encoder}

My approach consists of the feature extraction stage, the sequence decode stage, and the final character mapping state.
In the feature extraction stage, it is intuitive to use the Bert encoder from the pixel level of the image.
However, due to the computational resource limit and the time limit,
It is wise to encode at sub-pixel level.
To be specific, it is practical to encode after the Residual Network extraction, then use the Bert model to encode the result.

The BERT objectives and the sequence Transformer architecture to predict the language tokens.

\par Given an unlabeled dataset $\Bbb X$, the BERT objective samples a sub-sequence $\Bbb S \in \{C\}$,
$C$ represents all possible tokens, and such that for each index $i \in \Bbb S$,
there is independent probability $15\%$ of appearing in $\Bbb S$,
name the token $M$ as the BERT mask.
As equation (1), train the language model by minimizing the BERT objective of the "masked" elements $x_M$
conditioned on the "unmasked" ones $x_{\left[1,n\right]\backslash M}$.

\begin{equation}
  \lambda = \mathop{\Bbb E}\limits_{x\sim\Bbb X} \mathop{\Bbb E}\limits_{M} \sum_{i\in S}{\bigl[-\log{p(x_i\mid x_{\left[1,n \right]\backslash M})\bigr]}}
\end{equation}

\par The transformer decoder takes the image pixels and meta characters sequence $x_1,\cdots,x_n$ and produces a $d$-dimensional
embedding for each position.

For the per-pixel image classification approach, for every RGB channel of pixels in an image,
each pixel had its pixel-channel code, ranges from 0x00 to 0xff for different colors.
Thus, taking these pixels in an image is identical to ASCII characters in a document.
Generality speaking, the performance of the pre-trained language models achieved in the document classification tasks, can be transferred to image classification naturally.

\par Recall that Kim~\cite{kim2014convolutional} had proved that unsupervised pre-trained language model $word2vec$ and CNN outperformed many other machine learning algorithms,
\eg support vector machines and conditional random field, in many datasets such as movie reviews, Stanford sentiment treebank, and the TREC question.
Cui \etal's pre-trained language model, namely RBT3, RBTL3, and RoBERTa-wwm-ext-large, had improved performances over many other machine learning algorithms.
BERT and RoBERTa-wwm-ext-large models both achieved an f1-score of 97.8\% in the THUCNews dataset, which contains 65 thousands of news in 10 domains.

\par From Table (1), by combining the pre-trained language model with fully connected layer as the document classification model,
the test accuracy exceeds the other popular machine learning algorithms.
Therefore, the pixel channels of an image can be properly represented by these pre-trained language models.


\begin{equation}
  \sigma(\textbf{z})_j = {e^{\textbf{z}_j} \over {\sum_{k=1}^{\textit{K}} e^{\textbf{z}_k}}}, j=1,\cdots,\textit{K}.
\end{equation}

%-------------------------------------------------------------------------


\section{Experiments and Results}


\subsection{Training}

The Baidu Street View character dataset contains 188560 annotated images and randomly choose 892 images as the test set from the total images .

As Cui \etal~\cite{cui2020revisiting} reported that the training of the RBT3 language model was based on Devlin \etal's model,
moreover, the pre-trained Chinese language models use extra 5 billion training tokens comparing to Devlin \etal's 0.4 billion tokens.
I use a batch size of 32 and train for $10$ epochs using AdamW with $\beta_1=0.99$, $\beta_2=0.999$, and weight decay of $10^{-8}$.
I set the learning rate to $10^{-4}$ and no warmed up or decay are used.

\par The training time for each epoch batch size is 136ms, comparing to the 20ms for one epoch in the legacy model without the Bert encoding layer.
Thus, the time cost for the training increased 5.8 times of before.

\par The experiment was performed on a Google Cloud Nvidia v100, with 32GB of CPU RAM, and with 16GB of GPU memory,
which can provide 125 tera-flops of computation capability.

\subsection{Analysis}

For the Bert language model layer, the Table (1) showed that the accuracy of the test set has improved compare to the model without Bert layer.
For the Bert language model, there are different number of layers ranging from $3\sim24$, and for each layer,
there are different hidden numbers of neurons for each layer ranging from $768\sim1024$.
As for the purpose of this paper is to show that the Bert model enhances the accuracy in the Chinese character recognition task,
I use the least layer and least number of neuron for the bert language model, named RBT3 provided by Cui \etal,
which was fully pre-trained using the corpus with 4.5 billions of Chinese tokens.
The model can be easily modified and trained on other languages, \eg English,
and take less time to train for the less amount of the character types comparing to
4095 different type in Chinese.


\begin{table}
\begin{center}
\begin{tabular}{|l|c|c|c|r|}
\hline
Model & Dim & Dataset& Epochs & Accuracy \\
\hline\hline
ResNet      & 18 &  ICDAR2015     & 50 &	23.67\% \\
ResNet      & 18 &  Baidu2019	  & 30 &	56.25\% \\
ResBertNet  & 18+3 & Baidu2019    & 72	&	79.82\% \\
\hline
\end{tabular}
\end{center}
\caption{Comparision of accuracy of character detection models on the Baidu Street View 2019 dataset and the ICDAR2015 dataset.
The ResBertNet uses 18 layers of Residual Network and 3 layers of Transformers.}
\end{table}


\subsection{Discussion}

From Table (1) we can see that, on the ICDAR2015~\cite{karatzas2015icdar} dataset, the ResNet text recognition model has accuracy of 23.67\%,
though the ICDAR2015 test set only contains 92 different English characters, numbers and punctuations,
the ResNet model cannot achieve a satisfactory result after 50 epochs.
From the training logs, I can see that the model has converged on a optimal point that after the 30th epoch,
and has no improvement in accuracy afterwards.
It shows that Street Scene View text recognition task is a hard problem and there is no easy way out.
Moreover, the Baidu Chinese street text view dataset contains 4095 different character types.
Compare to the accuracy of 56.25\% on the Baidu2019 street view text dataset, our models have preferable better results.
The reason that the model's outstanding performance lies in the large pre-trained data for BERT, on top of that fine-tuned by RoBERTa, and use of extra language corpus of
$5.4$ billions of tokens of wiki data and other resources.
The transformers in the pre-trained language models use multiple layers for representing images and may be used in other Computer Vision task, \eg object detection, gesture prediction.


\section{Conclusion}

This paper proposes a novel idea by using pre-trained language models for Chinese scene text detection and recognition.
The finding improves the accuracy in OCR for Chinese street scene, and might improve the accuracy of other language scene detection and recognition.
Tests showed that the proposed model outperforms the ResNet model without augmentation on the image dataset,
the model achieved accuracy of $79.82\%$ on the Baidu 2019 street view text recognition set.



{\small
\bibliographystyle{ieee_fullname}
%\bibliography{cpvr_detrec}
\begin{thebibliography}{10}

  \bibitem{lee2009convolutional}
Honglak Lee, Roger Grosse, Rajesh Ranganath, and Andrew~Y Ng.
\newblock Convolutional deep belief networks for scalable unsupervised learning
  of hierarchical representations.
\newblock In {\em Proceedings of the 26th annual international conference on
  machine learning}, pages 609--616, 2009.


\bibitem{nair2010rectified}
Vinod Nair and Geoffrey~E Hinton.
\newblock Rectified linear units improve restricted boltzmann machines.
\newblock In {\em ICML}, 2010.


\bibitem{mikolov2013distributed}
Tomas Mikolov, Ilya Sutskever, Kai Chen, Greg~S Corrado, and Jeff Dean.
\newblock Distributed representations of words and phrases and their
  compositionality.
\newblock In {\em Advances in neural information processing systems}, pages
  3111--3119, 2013.


\bibitem{devlin2019bert}
Jacob Devlin, Ming-Wei Chang, Kenton Lee, and Kristina Toutanova.
\newblock Bert: Pre-training of deep bidirectional transformers for language
  understanding.
\newblock In {\em NAACL-HLT (1)}, 2019.


\bibitem{vincent2008extracting}
Pascal Vincent, Hugo Larochelle, Yoshua Bengio, and Pierre-Antoine Manzagol.
\newblock Extracting and composing robust features with denoising autoencoders.
\newblock In {\em Proceedings of the 25th international conference on Machine
  learning}, pages 1096--1103, 2008.


\bibitem{wang2012end}
Tao Wang, David~J Wu, Adam Coates, and Andrew~Y Ng.
\newblock End-to-end text recognition with convolutional neural networks.
\newblock In {\em Proceedings of the 21st international conference on pattern
  recognition (ICPR2012)}, pages 3304--3308. IEEE, 2012.


\bibitem{lucas2003icdar}
Simon~M Lucas, Alex Panaretos, Luis Sosa, Anthony Tang, Shirley Wong, and
  Robert Young.
\newblock Icdar 2003 robust reading competitions.
\newblock In {\em Seventh International Conference on Document Analysis and
  Recognition, 2003. Proceedings.}, pages 682--687. Citeseer, 2003.


\bibitem{wang2011end}
Kai Wang, Boris Babenko, and Serge Belongie.
\newblock End-to-end scene text recognition.
\newblock In {\em 2011 International Conference on Computer Vision}, pages
  1457--1464. IEEE, 2011.



\bibitem{lecun1990handwritten}
Yann LeCun, Bernhard~E Boser, John~S Denker, Donnie Henderson, Richard~E
  Howard, Wayne~E Hubbard, and Lawrence~D Jackel.
\newblock Handwritten digit recognition with a back-propagation network.
\newblock In {\em Advances in neural information processing systems}, pages
  396--404, 1990.


\bibitem{liao2020real}
Minghui Liao, Zhaoyi Wan, Cong Yao, Kai Chen, and Xiang Bai.
\newblock Real-time scene text detection with differentiable binarization.
\newblock In {\em AAAI}, pages 11474--11481, 2020.


\bibitem{yao2012detecting}
Cong Yao, Xiang Bai, Wenyu Liu, Yi~Ma, and Zhuowen Tu.
\newblock Detecting texts of arbitrary orientations in natural images.
\newblock In {\em 2012 IEEE conference on computer vision and pattern
  recognition}, pages 1083--1090. IEEE, 2012.

  \bibitem{liu2019curved}
Yuliang Liu, Lianwen Jin, Shuaitao Zhang, Canjie Luo, and Sheng Zhang.
\newblock Curved scene text detection via transverse and longitudinal sequence
  connection.
\newblock {\em Pattern Recognition}, 90:337--345, 2019.


\bibitem{chen2020generative}
Mark Chen, Alec Radford, Rewon Child, Jeff Wu, Heewoo Jun, Prafulla Dhariwal,
  David Luan, and Ilya Sutskever.
\newblock Generative pretraining from pixels.
\newblock In {\em Proceedings of the 37th International Conference on Machine
  Learning}, 2020.



\bibitem{wang2019development}
Wei Wang, Yujing Yang, Xin Wang, Weizheng Wang, and Ji~Li.
\newblock Development of convolutional neural network and its application in
  image classification: a survey.
\newblock {\em Optical Engineering}, 58(4):040901, 2019.

  \bibitem{krizhevsky2012imagenet}
Alex Krizhevsky, Ilya Sutskever, and Geoffrey~E Hinton.
\newblock Imagenet classification with deep convolutional neural networks.
\newblock In {\em Advances in neural information processing systems}, pages
  1097--1105, 2012.


\bibitem{netzer2011reading}
Yuval Netzer, Tao Wang, Adam Coates, Alessandro Bissacco, Bo~Wu, and Andrew~Y
  Ng.
\newblock Reading digits in natural images with unsupervised feature learning.
\newblock 2011.


\bibitem{zeiler2014visualizing}
Matthew~D Zeiler and Rob Fergus.
\newblock Visualizing and understanding convolutional networks.
\newblock In {\em European conference on computer vision}, pages 818--833.
  Springer, 2014.

  \bibitem{simonyan2014very}
Karen Simonyan and Andrew Zisserman.
\newblock Very deep convolutional networks for large-scale image recognition.
\newblock San Diego, CA, United states, 2015.
\newblock Convnet;Convolution filters;Convolutional networks;Localisation;Prior
  arts;State of the art;Visual representations;.

\bibitem{szegedy2015going}
Christian Szegedy, Wei Liu, Yangqing Jia, Pierre Sermanet, Scott Reed, Dragomir
  Anguelov, Dumitru Erhan, Vincent Vanhoucke, and Andrew Rabinovich.
\newblock Going deeper with convolutions.
\newblock In {\em Proceedings of the IEEE conference on computer vision and
  pattern recognition}, pages 1--9, 2015.


\bibitem{he2016deep}
Kaiming He, Xiangyu Zhang, Shaoqing Ren, and Jian Sun.
\newblock Deep residual learning for image recognition.
\newblock In {\em Proceedings of the IEEE conference on computer vision and
  pattern recognition}, pages 770--778, 2016.



\bibitem{xie2017aggregated}
Saining Xie, Ross Girshick, Piotr Doll{\'a}r, Zhuowen Tu, and Kaiming He.
\newblock Aggregated residual transformations for deep neural networks.
\newblock In {\em Proceedings of the IEEE conference on computer vision and
  pattern recognition}, pages 1492--1500, 2017.



\bibitem{szegedy2016inception}
Christian Szegedy, Sergey Ioffe, Vincent Vanhoucke, and Alexander~A. Alemi.
\newblock Inception-v4, inception-resnet and the impact of residual connections
  on learning.
\newblock pages 4278 -- 4284, San Francisco, CA, United states, 2017.
\newblock Classification tasks;Computational costs;Convolutional
  networks;Single frames;State-of-the-art performance;Test sets;Traditional
  architecture;.


\bibitem{huang2016deep}
Gao Huang, Yu~Sun, Zhuang Liu, Daniel Sedra, and Kilian~Q Weinberger.
\newblock Deep networks with stochastic depth.
\newblock In {\em European conference on computer vision}, pages 646--661.
  Springer, 2016.



\bibitem{lin2017feature}
Tsung-Yi Lin, Piotr Doll{\'a}r, Ross Girshick, Kaiming He, Bharath Hariharan,
  and Serge Belongie.
\newblock Feature pyramid networks for object detection.
\newblock In {\em Proceedings of the IEEE conference on computer vision and
  pattern recognition}, pages 2117--2125, 2017.



\bibitem{lu2007survey}
Dengsheng Lu and Qihao Weng.
\newblock A survey of image classification methods and techniques for improving
  classification performance.
\newblock {\em International journal of Remote sensing}, 28(5):823--870, 2007.


\bibitem{cui2020revisiting}
Yiming Cui, Wanxiang Che, Ting Liu, Bing Qin, Shijin Wang, and Guoping Hu.
\newblock Revisiting pre-trained models for chinese natural language
  processing.
\newblock In {\em Findings of EMNLP}. Association for Computational
  Linguistics, 2020.


\bibitem{he2017fine}
Xiangteng He and Yuxin Peng.
\newblock Fine-grained image classification via combining vision and language.
\newblock In {\em Proceedings of the IEEE Conference on Computer Vision and
  Pattern Recognition}, pages 5994--6002, 2017.



\bibitem{reed2016learning}
Scott Reed, Zeynep Akata, Honglak Lee, and Bernt Schiele.
\newblock Learning deep representations of fine-grained visual descriptions.
\newblock In {\em Proceedings of the IEEE Conference on Computer Vision and
  Pattern Recognition}, pages 49--58, 2016.


\bibitem{xie2019scene}
Enze Xie, Yuhang Zang, Shuai Shao, Gang Yu, Cong Yao, and Guangyao Li.
\newblock Scene text detection with supervised pyramid context network.
\newblock In {\em Proceedings of the AAAI Conference on Artificial
  Intelligence}, volume~33, pages 9038--9045, 2019.




\bibitem{tian2019learning}
Zhuotao Tian, Michelle Shu, Pengyuan Lyu, Ruiyu Li, Chao Zhou, Xiaoyong Shen,
  and Jiaya Jia.
\newblock Learning shape-aware embedding for scene text detection.
\newblock In {\em Proceedings of the IEEE Conference on Computer Vision and
  Pattern Recognition}, pages 4234--4243, 2019.

\bibitem{kim2014convolutional}
Yoon Kim.
\newblock Convolutional neural networks for sentence classification.
\newblock pages 1746 -- 1751, Doha, Qatar, 2014.
\newblock Classification tasks;Convolutional neural
  network;Hyper-parameter;Learning tasks;Question classification;Sentence
  classifications;Simple modifications;State of the art;.


\bibitem{karatzas2015icdar}
Dimosthenis Karatzas, Lluis Gomez-Bigorda, Anguelos Nicolaou, Suman Ghosh,
  Andrew Bagdanov, Masakazu Iwamura, Jiri Matas, Lukas Neumann,
  Vijay~Ramaseshan Chandrasekhar, Shijian Lu, et~al.
\newblock Icdar 2015 competition on robust reading.
\newblock In {\em 2015 13th International Conference on Document Analysis and
  Recognition (ICDAR)}, pages 1156--1160. IEEE, 2015.


\end{thebibliography}

}

\end{document}
