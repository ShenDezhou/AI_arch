% This version of CVPR template is provided by Ming-Ming Cheng.
% Please leave an issue if you found a bug:
% https://github.com/MCG-NKU/CVPR_Template.

\documentclass[review]{cvpr}
%\documentclass[final]{cvpr}

\usepackage{times}
\usepackage{epsfig}
\usepackage{graphicx}
\usepackage{amsmath}
\usepackage{amssymb}
% Include other packages here, before hyperref.
\usepackage[linesnumbered,lined,boxed,commentsnumbered]{algorithm2e}

% If you comment hyperref and then uncomment it, you should delete
% egpaper.aux before re-running latex.  (Or just hit 'q' on the first latex
% run, let it finish, and you should be clear).
\usepackage[pagebackref=true,breaklinks=true,colorlinks,bookmarks=false]{hyperref}


\def\cvprPaperID{8888} % *** Enter the CVPR Paper ID here
\def\confYear{CVPR 2021}
%\setcounter{page}{4321} % For final version only


\begin{document}

%%%%%%%%% TITLE
\title{Visualize Non-Euclidean Actors Social Media via Joint Social Relationship Representations, Network Characteristics and Media Sentiments}

\author{Dezhou Shen\\
Department of Computer Science\\
Tsinghua University\\
Beijing, CN 100084\\
{\tt\small sdz15@mails.tsinghua.edu.cn}
% For a paper whose authors are all at the same institution,
% omit the following lines up until the closing ``}''.
% Additional authors and addresses can be added with ``\and'',
% just like the second author.
% To save space, use either the email address or home page, not both
}

\maketitle


%%%%%%%%% ABSTRACT
\begin{abstract}
  In film industry, visualizing stars social media for analyzing audience intention towards movies is essential to producers.
  However, non-Euclidean social media is hard to analyze, I propose a FC-GRU-CNN architecture for social media visualization.
  As an empirical case, social relationship representations, network characteristics and media sentiments are used as key factors in box-office prediction model.
  The model exploits long-term memory ability of GRU layer in long sequences of cast,
  and the mapping ability of CNN layer in retrieving all pairs shortest path feature of actors social relationship,
  the model accuracy is 14\% higher than Zhou \etal's model.
\end{abstract}

%%%%%%%%% BODY TEXT
\section{Introduction}

With the development of social networks, actors' influence on social media is increasing, and the influence of actors' consumption is spread through social networks,
and using long-term social networks data to study the relationships between all actors, directors and films remains a challenge.
It is easy to see that the relationship between the box-office and the director, actors, scriptwriters and producers is often non-linear, that is, it is difficult to use linear functions to build box-office prediction models.
By analyzing the social network characteristics of movie actors, sentiments of movies in Sina Weibo posts, all pairs shortest path, it is easy to get the priority of each characteristic by analysis of feature importance,
existing prediction algorithm only makes use of the film meta-data, lacks the study on the social network characteristics of the actors, though traditional explanation of the film box-office prediction algorithm is sufficing, however the prediction effect is not satisfactory.
This paper presents a model, named FC-GRU-CNN, for movie box-office prediction, combining characteristics of film meta-data, actors' social network measurement,
all nodes pair shortest path, social network text sentiments and actors' art characteristics, proposed model is 14\% higher in accuracy than Zhou \etal~\cite{zhou2015c} model.

%-------------------------------------------------------------------------

\section{Recent Work}

Film industry is a multi-billion-dollar industry, as China's box-office exceeding 6 billion dollars in 2018, reaching 6.09 billion dollars.
Predicting the acceptance and adoption of new films among the audience is a challenging topic.

\subsection{Linear Regression Model}

Zhu and Tang~\cite{zhu2019film} collected 13 features, such as search engine user data, social network fan number and movie meta-data of 20 films with more than 300 million box-office in 2017,
and constructed a partial least-squared model with a margin of error of 87.7\% and an average absolute error of 26.6\%.
Qiu and Tang~\cite{qiu2018microblog} used the 10 movie Sina Weibo texts and movie reviews of more than 1 billion at the box-office in the Chinese film market in 2017 to calculate the web index and the film reviews to predict the box-office.

\subsection{Logistic Regression Model}

Some researchers use social network text sentiments to predict movie box-office.
Asur and Huberman~\cite{asur2010predicting} collected 2.89 million texts related to 1.2 million users on Twitter in connection with 24 films, and concluded that the correlation coefficient $R^2$ was between 0.92 and 0.97 using the daily postings in the first seven days of release.
The film box-office forecast logic regression model with the distribution of social network posts was 0.97.
Jain~\cite{jain2013prediction} collected 4800 Twitter texts of the movie name in the SNAP2009 (Stanford) dataset 2 weeks before release and 4 weeks after release.
Trained an sentiments al four-category model based on an 8-grammar model with an accuracy rate of 64.4\%.
Eight films released in 2012 as test sets, using the sentiments al positive-negative ratio to predict the accuracy of the box-office profit model by 50\%.
Chi \etal~\cite{chi2019does} collected reviews from 150 films released in 2017 and found that the number of reviews was positively related to the box-office.
online reviews were not correlated with box-office. and the number of reviews was significantly positive lying to the first week of the film's box-office, and the relevance gradually diminished.
Josh \etal~\cite{joshi2010movie} collected the film review text of the film released by MetaCritic from 2005 to 2009, using film meta-data, n-gram model, part-of-speech ngram model and dependency characteristics, to predict the first week box-office,
first week screen number and weekly revenue per screen, the determine of coefficient of the Elastic Net linear model is 67.1\%.

\subsection{Support Vector Regression and Multi-Layer Neural Network Model}

Jiang and Wang~\cite{jiang2018predicting} collected 34 high-grossing films from 2015 to 2017, using 10 features, such as film meta-data, director awards, and actor awards, and used voting feature selection algorithms to select features, training SVR (supporting vector regression) box-office and scoring prediction models.
Quader \etal~\cite{quader2017machine} collected 755 film reviews from four film review sites from 2012 to 2015, including film ratings, MPAA categories, director's total box-office, total box-office, actor total box-office, release time, budget, screen numbers, user reviews, reviews, and the sentiments al characteristics of review texts,
training SVR, MLP model, with an accuracy of about 44.4\%.

\subsection{Text Classification using Deep Learning}

Zhou\etal proposed a C-LSTM text classification model, using single-layer CNN for extracting the high-order dimensional representation of text, and then connected to LSTM layers for sentence representation, accuracy of the model is 87.8\%.

\subsection{Summary}

From the social network to predict the movie box-office, feature selection includes the selection of user behavior characteristics, social network time characteristics, text sentiments al characteristics, classifiers more use of choice logic regression, support vector machine and neural network to predict the box-office and comments.

%-------------------------------------------------------------------------

\section{Characteristics Correlation Analysis}

Correlation analysis of social network characteristics and the box-office is carried,
and discussed the practical significance of the interaction between the characteristics and the box-office.
In the linear regression analysis, the positive correlation characteristics associated with the box-office are:
\begin{itemize}
\item {\bf Betweenness}: the higher the number of actors, the higher the box-office gets.
\item {\bf Closeness}: The higher the closeness of the actors, the higher the box-office gets.
\item {\bf Followers}: The more other actors you pay attention to, the higher the box-office gets.
\item {\bf Number of posts}: the more actors, the higher the box-office gets.
\item {\bf Average post interval}: the greater the average hair interval, the higher the box-office gets.
\item {\bf Average characters length of each posts}: the longer the number of characters, the higher the box-office, but the degree of relevance is not obvious.
\item {\bf Number of retweets}: the more retweet posts, the higher the box-office gets.
\item {\bf Box-office history}: the higher the box-office receipts of actors, the higher the box-office attendance of participating in the film gets.
\end{itemize}

And associated with the box-office, the negative correlation characteristics are:
\begin{itemize}
\item {\bf Number of fans}: the more fans there are, the lower the box-office gets.
\item {\bf Number of movie-related Sina Weibo posts}: the larger number the actors' mentioned the movie in Sina Weibo, the lower the box-office gets.
\item {\bf Co-occurrence of posts}: The more multiple actors co-appears in Sina Weibo posts, the lower the box-office gets.
\end{itemize}

\par Then, I calculated Pearson correlation of average box-office per movie and actors social network characteristics, the post and retweet correlation is $-0.108\%$, social media Activity correlation is $-3.203\%$, Co-occurrence correlation is $15.08\%$.
First of all, the actors' average box-office per movie is clearly positively related to the influence of the actors, that is, the higher number of co-occurrence in Sina Weibo, the higher average box-office per movie.
Secondly, the actors Sina Weibo activity and the average box-office per movie showed a slight negative correlation, that is, the number of actors' followers, the number of tweets, the interval between the posts, the length of posts, the number of movie-related posts showed negative impact on average box-office per movie.
Finally, the number of actors retweets and tweets is not related to average box-office per movie, thus they are independent.
Then I use Sina Weibo text sentiments and actor social network measurement characteristics as features, then trained a decision tree to fit the movie box-office, training set determination of coefficient is 0.9858, and the test set determination of coefficient is 0.9605.
Actors social network meta-data decision tree visualization is illustrated as Figure 2.

\par \noindent Definition 1 (\textbf{Gini Impurity}) A measure of how often randomly selected elements are incorrectly marked from a collection is a common CART algorithm evaluation indicator.
If it is randomly marked based on the distribution of the label in a subset.
Gini impurities can be divided by multiplying the probability of the item by the probability that it is misplaced.
When in all conditions a node belong to a single category, it reaches its minimum value of zero.
In summary, the decision tree structure shows that the influence of the actors in the social network is the main factor affecting the box-office results, followed by the movie-related Sina Weibo sentiments and the film meta-data is the secondary determinant,
while the social network characteristics, movie-related Sina Weibo sentiment, film meta-data are the general determinants.


\section{Representation Learning for Social Relationship, Network Characteristics And Art Contribution}

\par \noindent Definition 2 (\textbf{Network Embedding})  A network $\Bbb G=(\Bbb V,\Bbb E)$, $\Bbb V$ represents a set of nodes consisting of $n$ nodes, and $\Bbb E$ represents an edge set of $m$ edges.
For each edge $e\in \Bbb E$, the ordered pair is $e=(u,v)$, $u,v \in \Bbb V$ and the weight is $A_{ij}$.

A network can be represented as an adjacent matrix, $A\in \Bbb R^{(n*n)}$, and the goal of network embedding is to learn an all pairs representation, $U\in \Bbb R^{(n*m)}$, $m < n$, where $m$ is an embedding dimension.\par
\par \noindent Definition 3 (\textbf{Shortest Path Network Embedding})  In a network represented as a $\Bbb G=(\Bbb V,\Bbb E)$, $\Bbb V$ represents a node set of $n$ nodes, and $\Bbb E$ represents an edge set of $m$ edges.
For each edge $e\in \Bbb E$, the ordered pair is $e=(u,v)$, $u,v \in \Bbb V$ and the weight is $A_{ij}$.

\par A network can be represented as an adjacent matrix, $A\in \Bbb R^{(n*n)}$, and the goal of the shortest path network embedding is to learn an all pairs, shortest path representation, $U\in \Bbb R^{(n*n)}$, where $n$ is an embedding dimension.
For the task of box-office forecasting, because of the significant impact the actors on the box-office, in addition to the measurement of the film meta-data and the sentiments of the movie social media, the actors' contribution is required to compute by leveraging the movie participants actors list.
A film, led by several actors, obtained the names of 7,369 actors who participated in films between 2011 and 2015 are collected as an actors' dictionary, and then used the shortest path features of the actors' social network.
Combining social network measurement features, the shortest path network embedding, and the art features to construct the actor representation vector.
For actors who have no accounts on social networks, marked as unknown, and initialized with zero vectors.
Actor representation vectors include network characteristics of 10 dimensions, shortest path features of 8380 dimensions, art features of 1 dimension.
In total, 10-dimension features include fans number, betweenness, closeness, followers, posts number, average posts interval, average word count per post, retweet number, movie-related post number, co-occurrence number, shortest path with 8380 dimensions.

\section{Machine Learning Classifier}

Using the features extracted, the characteristics of participating in the calculation are film meta-data and movie sentiment:
the year of release, the number of positive sentiment, the number of negative sentiment, the total number of Sina Weibo posts,
the positive and negative ratio of sentiment, and the training of machine learning classifiers.

Machine learning methods such as Linear Regression, SVR, MLP, Logistic Regression, and Naive Bayes cannot converge on the data set,
possibly because the sentiments on actors' social networks does not represent the commercial benefits of the film.
Actors, as creators of films, have a role to play in the box-office, but lack of other important features that also have an significant impact on the box-office,
so machine learning fitting process can't finish in given time.
Using the CART decision tree regression algorithm to make box-office prediction for the test set film, 
the experimental results show that the prediction model has a determination coefficient of 96.05\% on the test set and 98.58\% on the training set, which is 2.87\% higher than the Qiu and Tang~\cite{qiu2018microblog} model,
and the average absolute deviation of the model is 0.1285 million, about 28.56\% of the average box-office.
The explanatory of the decision tree prediction model is more intuitive and more interpretable than linear regression, logical regression, deep neural network and other models,
according to the prediction model structure schematic of the movie box-office,
the characteristics can be sorted according to the importance, the social network structure characteristics are the most important,
the social network text sentimental characteristics, the film meta-data are important, and the social network characteristics are generally important.
Compared to Qiu and Tang~\cite{qiu2018microblog}'s logistic regression model on sentiments and twitter index algorithm's 95.71\% determination of coefficient,
the proposed CART model in this paper is 98.58\% which has a 2.87\% improvement.

\section{FC-GRU-CNN Box-Office Prediction Model}


\begin{figure*}
\begin{center}
  \fbox{\rule{0pt}{2in} \includegraphics[width=0.9\linewidth]{boxofficenet.pdf}}
\end{center}
   \caption{FC-GRU-CNN (box-office net) forecasting model structure.}
\label{fig:short}
\end{figure*}

Look at the box-office distribution of 1,296 films released between 2011 and 2015,
Consider box-office forecasts as a binary classification task.
In order to make the data evenly distributed, the film data was pre-processed by using the median box-office of 2.635 million as the dividing line.
So for the movies with box-office less than 2.635 million are classified as A-class, movies with box-office more than 2.635 million are classified as B-class.
In the dataset division, the data was randomly selected using the $80-20$ segmentation ratio, and the training set of 1036 films and the test set of 260 movies were collected.
Using the actors' social network representation, the actor is quantified, that is, 
the fusion of social network features, the shortest path characteristics and the actors' art characteristics.
For actors without social network data, use zero-value initialization in the actor feature matrix for the unknown characteristics.
As shown in Figure 1, the model are divided into three parts:

\begin{itemize}
\item The first part is Fully Connected layer, its input feature is the film meta-data and movie-related Sina Weibo sentiments, in total of 5 dimensions.
\item The second part is the GRU layer, the input feature is the actor social network measurement characteristics and art characteristics, the film cast is a long list,
the longest 225 actors, for less than 225 actors using the front-end padding, the GRU layer is to solve the long-term memory loss of the structure.
\item The last part is the 5 CNN layers, input for the actors' social network shortest path feature, similarly, the film cast is a long list,
input less than 225 dimensions is fulfilled by padding, CNN has a good ability to capture multi-dimensional features, and use of Max-pooling layer to solve the problem of multi-dimension feature extraction.
\end{itemize}
Finally, the three parts are merged into one layer and then connects to a Softmax layer, and the L2 regularization method is adopted to avoid overfitting.

In summary, randomly choose 10 movies as test set, as shown in Table 1.
The accuracy of the binary classification model in training set reached 99.61\%, the accuracy in the test set is 75\%.
Compared to Zhou \etal model, see Table 2, the proposed FC-GRU-CNN model converges in the training set and achieves better prediction results in the test set.
Therefore it can provide decision-making advice in the task of selecting suitable actors for to-be-casted film,
and in box-office prediction task, the model can provide reasonable prediction in the market.

\begin{table}
\begin{center}
\begin{tabular}{|l|c|c|}
\hline
Movie& Score &BoxOffice\\
\hline\hline
72 hours of sword theft & 1 & 108 \\
Ballera's Little Magic Fairy& 1 & 1916\\
It's us. &1& 28\\
Twin Spirits &1& 310\\
The dinner is crazy too& 1& 4381\\
Basement Horror& 0& 704\\
Tartar Oil Flower Romance& 0& 1\\
Breaking five& 0& 1\\
Kidnapping the door dog& 1& 1\\
Beijing Love Story& 1& 40555\\
\hline
\end{tabular}
\end{center}
\caption{box-office binary classification predictions in 10 movie testset.}
\end{table}



\begin{table}
\begin{center}
\begin{tabular}{|l|c|}
\hline
Model & F1-score \\
\hline\hline
Zhou \etal  & 61.54\% \\
This paper & 75.0\% \\
\hline
\end{tabular}
\end{center}
\caption{Performance comparison on different model.}
\end{table}

\section{Conclusion}

According to the box-office prediction model, joint actors' Sina Weibo sentiments, social relationship and network characteristics,
this paper discussed relationship between the movie meta-data, Sina Weibo text sentiments, social relationship representations, network characteristics and actors' art contribution.
Then I proposed a FC-GRU-CNN box-office prediction model, compared to Zhou \etal model, accuracy improves 14\%,
and tested in a dataset of 1,296 movies released in China mainland collected from 2011 to 2015, proposed model has an accuracy of 75\%.

{\small
\bibliographystyle{ieee_fullname}
\bibliography{cpvr_vision}
}

\end{document}
